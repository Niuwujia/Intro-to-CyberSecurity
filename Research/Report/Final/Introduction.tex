在过去的几十年里,随着信息技术的迅速发展和广泛应用,个人和组织拥有越来越多的敏感数据,例如医疗记录、金融数据等。这引发了对隐私和数据保护的关注,促使研究人员开发了各种隐私保护计算方案。\\

在许多场景中,两个或多个参与方可能希望计算它们的数据之间的交集,但又不希望共享原始数据,以保护隐私。例如,在医疗研究中,不同的医院可能希望比较其患者群体的共同特征,而不共享具体的患者数据;多个银行想要共享客户的风险信息,就可以使用PSI技术找出存在风险的客户,而不泄露其他客户的信息;运营商和金融机构想要对齐共同的客户群体,然后进行联合建模和定点营销,就可以使用PSI技术在不泄露客户隐私的前提下实现数据探查和模型训练等等。因此,出于隐私保护的考虑,研究人员开始研究如何在不泄露个体数据的情况下进行集合求交,即这篇文章中讨论的隐私集合求交问题,PSI.\\

2004年,Freedman等\cite{freedman2004efficient}首次提出了隐私集合交集协议问题,分别构造了基于标准模型下的半诚实环境适用协议和基于随机预言机模型的恶意环境适用协议。2005年,Kissner等\cite{kissner2005privacy}提出了多方集合协议。2010年,De Cristofaro等\cite{de2010practical}提出了一个具有线性复杂度授权的PSI协议(Authorized PSI, APSI)。2012年,Huang等\cite{huang2012private}将PSI问题转化为布尔函数的电路问题。2014年,Benny Pinkas等\cite{pinkas2014faster}提出了基于OT扩展的高效PSI协议。2016年,Vladimir Kolesnikov等\cite{kolesnikov2016efficient}设计的基于batched OT 扩展传输和布谷鸟哈希实现了更高效的隐私集合求交方案,成为性能上最接近朴素哈希求交技术的隐私集合求交方案。2017年,Chen等\cite{chen2017fast}提出了基于同态加密的PSI协议。我们研究分析了上述文章中提出的协议的思想和具体流程,分析了其方法的优点和不足。\\

隐私集合求交的主要挑战是如何在保证安全性和效率的同时,最小化通信开销和计算复杂度。APSI是一种隐私集合求交的变体,它要求参与方的集合中的元素必须由一个可信的第三方(如证书颁发机构)进行授权。APSI可以保证只有授权的元素才能参与交集计算,而未授权的元素不会泄露给对方。De Cristofaro等人在Practical Private Set Intersection Protocols with Linear Complexity一文\cite{de2010practical}中对以往的APSI协议进行改进,提出了高效的PSI协议。我们编写代码实现了他们的协议。\\

但是,De Cristofaro等人在论文中\cite{de2010practical}提及,他们设计的PSI协议的其中一个缺点是不知如何转为对应的APSI协议。为此,在De Cristofaro等人的基础上,进行一系列数学推导,我们设计了基于第三方证书的APSI协议,并给出了我们的编程实现,并进行了一系列测试。结果表明,我们的协议能够不仅能够完成基本的PSI的功能,还能对客户端的元素进行验证,防止客户端传输虚假的(携带无效证书)的元素。