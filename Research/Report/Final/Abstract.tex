隐私集合求交(Private Set Intersection, PSI)是一种安全多方计算的密码学技术,它允许参与的双方在不泄露除交集之外的任何额外信息的情况下,计算各自持有的集合的交集。这种协议可以用于各种隐私保护场景,例如数据共享、数据分析和身份验证等。在这篇文章中,我们将介绍隐私集合求交领域的背景和相关工作,并对De Cristofaro等人提出的PSI协议进行编程实现。在此基础上,我们还对De Cristofaro等人的PSI协议改进为对应的APSI(Authorized PSI, 授权的PSI)版本。我们提出并设计了三个版本的基于Blind RSA的 APSI 协议,分别适用于不同的时间复杂度和安全性要求。此外我们还对协议的正确性、安全性和效率进行了理论分析,并在实际环境中进行了性能测试,与其他现有的 PSI 协议进行了比较。我们的结果分析表明,基于Blind RSA 的 APSI 协议具有较高的计算效率和通信效率,且能够抵抗非法用户的恶意攻击,达到隐私保护和安全防御的目的。